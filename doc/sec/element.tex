\chapter{Elements}
\label{ch:elements}
\todo{Talk about ``affinities'' rather than ``resistances''?}

Elements participate in calculating resistances between skills and entities.
Elements also have status effects associated with them,
and generally entities of a particular element have common strengths or weaknesses.
(Not always, however!)

\todo{Actually implement status effects, and nut out for sure what status
effects are associated with what elements.}
\todo{Work out what elements have what strenghts and weaknesses.}

There are three classes of elements:
\begin{itemize}
    \item \hyperref[sec:element_neutral]{neutral}: the default element, |element.neutral|,
        which encompasses everything that otherwise doesn't belong in an element
    \item \hyperref[sec:element_primary]{primary}: the building block elements
    \item \hyperref[sec:element_secondary]{secondary}: specialised elements,
        these are composed from combinations of primary elements
        and sharing their effects to a degree
\end{itemize}

For an element named |X|,
we can spell that element in our configuration files as ``|element.X|''.

\section{Neutral}
\label{sec:element_neutral}

The neutral element, spelt |element.neutral|,
is the catch-all for anything that doesn't have an element.
For example, basica attacks often have no element.
It's also the element for many status effects,
unless they are explicitly associated with a different element;
for example, Sleep would be a neutral element.
\note{Status effects have not yet been implemented.}

It is possible to grant resistance to the neutral element for entities,
but it isn't recommended unless careful thought is given due to balance issues.
There are no particular attributes in common for entities with neutral element.

\section{Primary}
\label{sec:element_primary}

The primary elements are the building blocks for elemental affinities.
They correspond to the four classical elements
as well as the two ``moral'' elements, Light and Dark.

\begin{apidoc}[Primary elements][tbl:element_primary]{ll}
    \thead{Element} & \thead{Spelling} \\
    \midrule
    Fire  & |element.fire| \\
    Water & |element.water| \\
    Earth & |element.earth| \\
    Air   & |element.air| \\
    Light & |element.light| \\
    Dark  & |element.dark| \\
    \todo{Include associated status effects or associated stat changes in this table}
\end{apidoc}

\section{Secondary}
\label{sec:element_secondary}

These elements are formed from combinations of
\hyperref[sec:element_primary]{primary elements}.
At the moment no secondary is constituted by one of the moral elements, however.

Resistances for secondary elements are calculated in an interesting fashion.
Entities can have specific affinities to a secondary element,
and this affects their resistance to that element.
However, their resistances to the constituent primary elements are also considered.
For example, an entity's resistance for Ice is calculated based upon
not only their Ice resistance, but also their Water and Air resistances.
\todo{What is the exact formula?}

\begin{apidoc}[Secondary elements][tbl:element_secondary]{lll}
    \thead{Element} & \thead{Spelling} & \thead{Constituents} \\
    \midrule
    Ice       & |element.ice|       & Air, Water \\
    Lightning & |element.lightning| & Air, Fire \\
    Steam     & |element.steam|     & Fire, Water \\
    Life      & |element.life|      & Earth, Water \\
    Metal     & |element.metal|     & Earth, Fire \\
    Dust      & |element.sand|      & Air, Earth \\
    \todo{Finalise what we call the Dust element,
          and update \lstinline{element.sand} to match.}
    \todo{Include relevant status effects}
\end{apidoc}
