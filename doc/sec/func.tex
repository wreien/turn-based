\chapter{Helper Functions}
\label{ch:func}

These are extra functions that are made available for use
within user-facing Lua code.
Note that standard-library functions from |base| and |math| are also available.

The notation used here is of the form |func(arg1, arg2[, optional])|,
to represent a function |func|
which takes two mandatory arguments |arg1| and |arg2|
and one optional argument |optional|.

\section{General}
\label{sec:func_general}

\subsection{\lstinline{random([m [, n]])}}
\label{sec:func_general_random}

When called without arguments,
returns a uniformly distributed floating point value between 0 and 1.
When called with two integers |m| and |n|,
returns a uniformly-distributed integer in the range |[m, n]|.
A call to |random(n)| is equivalent to a call to |random(1, n)|.

This function is identical to |math.random|, but is here for consistency.

\subsection{\lstinline{randf([m [, n]])}}
\label{sec:func_general_randf}

Functions the same as |random| but for floating point values.
When called with two floating-point values |m| and |n|,
returns a uniformly distributed floating point value in the range \lstinline{[m, n)}.
A call to |randf(n)| is equivalent to a call to |randf(0.0, n)|,
and a call to |randf()| is equivalent to a call to |randf(0.0, 1.0)|.

\subsection{\lstinline{log(message)}}

Logs the message |message| to the process log for the turn.
This call is only valid when inside a context where logging makes sense,
such as in a skill's \hyperref[sec:skill_func]{\lstinline{perform} function}
(see \autoref{sec:skill_func}).

There are currently three sorts of messages that can be sent;
see \autoref{tbl:messages} for details. An example usage of this function is:
\begin{lstlisting}
    -- display an important message to the player; note the use of the ``..'' operator
    log(message.notify("IMPORTANT! 5 + 3 = " .. (5 + 3)))
\end{lstlisting}

\begin{apidoc}[Message types][tbl:messages]{ll}
    \thead{Function} & \thead{Description} \\
    \midrule
    |message.miss(entity)| & The current \hyperref[ch:skill]{skill} missed
        the specified \hyperref[ch:entities]{entity} \\
    |message.critical(entity)| & The current \hyperref[ch:skill]{skill} critically hit
        the specified \hyperref[ch:entities]{entity} \\
    |message.notify(string)| & Display the given |string| to the user in some manner \\
    \todo{Plain \lstinline{string} arguments as debugging aids ---
        or maybe another function \lstinline{message.debug(string)}?}
\end{apidoc}

The |message.notify| message type can also be useful for debugging.

\section{Skills}
\label{sec:func_skill}

Many of these functions take similar parameters.
See \autoref{tbl:func_skill_params} for an overview of the common parameters,
including their types and their description. If a parameter isn't listed there
it should be mentioned in the description of the particular function.

\begin{apidoc}[Standard skill function parameters][tbl:func_skill_params]{lll}
    \thead{Parameter} & \thead{Type} & \thead{Description} \\
    \midrule
    |s| & \hyperref[ch:skill]{skill} & The skill being used \\
    |source| & \hyperref[ch:entities]{entity} & The entity using the skill \\
    |target| & \hyperref[ch:entities]{entity} & The entity the skill is being used on \\
    |entity| & \hyperref[ch:entities]{entity} & A generic entity \\
\end{apidoc}

\subsection{\lstinline{skill.did_hit(s, source, target [, crit_difficulty])}}
\label{sec:func_skill_didhit}

Determines if a given skill managed to hit the target.
Apart from the usual parameters, described in \autoref{tbl:func_skill_params},
this function takes an optional parameter |crit_difficulty|,
which determines the difficulty of scoring a critical hit.
If not provided it defaults to 6.

This function will return one of three values:
\begin{itemize}[noitemsep]
    \item \textbf{0} $\implies$ The skill missed
    \item \textbf{1} $\implies$ The skill hit normally
    \item \textbf{2} $\implies$ The skill scored a critical hit
\end{itemize}
\todo{Return named enumerators?}

The function first tests as it hits at all, as a function of
the skill's |accuracy|, the source's |skill|, and the target's |evade|.
(See \autoref{tbl:skill_attrs} and \autoref{tbl:entity_stats}).
The probability of scoring a hit is
(where $P(x)$ returns true with probability $x\%$)
\[ P\left(\texttt{accuracy} + \texttt{skill} - \texttt{evade}\right) \]
If the skill hits, it then does another check,
which succeeds with chance
\[ P\left(\frac
    {\texttt{accuracy} + \texttt{skill} - \texttt{evade}}
    {\texttt{crit\_difficulty}}\right) \]
to determine if a critical hit was scored.

For example, if the skill's base accuracy is 85,
the source's skill is 10, and the target's evade is 5,
then the chance of scoring a hit is 90\%.
Assuming it hit, if \texttt{crit\_difficulty} is 6 (the default)
the chance of scoring a critical hit is 15\%.
As such, the overall chance of a critical hit in this situation is
$15\%\times 90\% = 13.5\%$.

\note{Is the default difficulty of critting too low or too high?}
\todo{\lstinline{skill} in this context is confusing; either find a better name for
    entity stats or rename the other kind of skill as \lstinline{action} or something.}

\subsection{\lstinline{skill.raw_damage(s, source, target)}}
\label{sec:func_skill_rawdamage}

Calculates the raw damage of a skill,
without taking into account modifiers,
thoguh it does account for status effects and traits.
For a description of the parameters, please see \autoref{tbl:func_skill_params}.

The function returns a positive floating point number
representing the base damage dealt. The damage is roughly calculated as
\[ \frac{\texttt{power}}{100}\times(4\times\texttt{attack} - 2\times\texttt{defense}) \]
but with some random jitter applied and clamped to positive values.

Note that the function takes into account the skill's
\hyperref[tbl:opt_method_spread]{method},
choosing the stats used for |attack| and |defense| appropriately.
It should not be used for skills with methods of |method.none| or |method.mixed|.

\subsection{\lstinline{skill.resistance(s, entity)}}
\label{sec:func_skill_resistance}

Calculates the resistance modifier for an entity for the given skill.
At this stage, the function only take into account
\hyperref[ch:elements]{elemental resistances}.
The parameters taken are described in \autoref{tbl:func_skill_params}.

The function returns a multiplier representing the resistance.
For example, a Fire element skill being used on
an entity with 20\% Fire resistance would return |0.8|.

\todo{Should this function take both \lstinline{source} and \lstinline{target}?}

\subsection{\lstinline{skill.default_perform(s, source, target)}}
\label{sec:func_skill_defaultperform}

Implements a generic |perform| function for skills.
It can be used to stamp out basic attacking skills with no other real forethought.
It has the same signature as the
\hyperref[sec:skill_func]{\lstinline{perform} function attribute}
required to define a skill, so it can be used like:
\begin{lstlisting}
    return {
        desc = "an example skill",
        -- \ldots other attributes

        perform = skill.default_perform
    }
\end{lstlisting}

The function, when called, will deal damage
according to the parameters for skills as described in \autoref{ch:skill}.
It doesn't, however, have any capabilities for perk- or level-specific modifications.

Note that this function can also be used as
a wrapped fallback for a skill where appropriate.
See \autoref{sec:skill_example} for an example of this.
