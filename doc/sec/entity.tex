\chapter{Entities}
\label{ch:entities}

Entities are currently defined using initialization files;
read some of the files in \texttt{data/entity/} for examples.
They're very much still a work in progress!

However, their interface into Lua code for use in skills is fairly set,
which is what this chapter will describe.
There will be minor changes as things progress,
but this should give you a good idea.

Entities have both attributes and functions.
The syntax for retrieving an attribute |attr| for an entity |e| is |e.attr|.
The syntax for calling a function |func| for an entity |e|
(sometimes with parameters |arg1|, |arg2|, etc.) is |e:func(arg1, arg2)|.
For example:

\begin{lstlisting}
    -- assuming `\texttt{e}' is an entity
    e.kind         -- what kind of entity this is
    e.level        -- what level the entity is
    e:drainHP(10)  -- do some damage
\end{lstlisting}

\section{Identification}
\label{sec:entity_ident}

There are three identifying traits for an entity.
At the top level is their ``kind'': this is the `class' of the entity,
or the overarching form for this set of entities.
For example, you might have a kind of ``golem'', or ``cat'', or ``human''.

Each ``kind'' of entity has a set of individual entity ``types'':
this is the `species' of entity, or a specific manifestation of a class.
So, you might have an ``aero golem'', or a ``blitz hyena'', or a ``bandit chieftain''.
This is what ultimately determines the base stats and abilities for an entity.

Finally, each entity also has a ``name''.
This uniquely identifies a given entity in the battlefield.
Usually this will be the type of entity appended with ``\#X'';
for example, ``blitz hyena \#2''
(if there was at least one other blitz hyena in the battle).
However, some special entities might have more interesting names;
for example, a bandit chieftain might be called ``Jeff''.

Each of these identifying traits can be retrieved with the following attributes:
\begin{apidoc}[Entity identification descriptors][tbl:entity_id]{lll}
    \thead{Attribute} & \thead{Type} & \thead{Description} \\
    \midrule
    |kind| & string & The ``kind'' of entity (e.g.\ cat) \\
    |type| & string & The ``type'' of entity (e.g.\ blitz hyena) \\
    |name| & string & The identifying nameoof the entity (e.g.\ blitz hyena \#2) \\
\end{apidoc}

\section{Data}
\label{sec:entity_data}

Entities also have a number of other attributes
representing useful information about them.
In approximate order of usefullnes, there is:
\begin{apidoc}[Entity attributes][tbl:entity_data]{lll}
    \thead{Attribute} & \thead{Type} & \thead{Description} \\
    \midrule
    |stats|      & \hyperref[sec:entity_stats]{stats} & The entity's stat block \\
    |is_dead|    & bool   & Whether the entity is dead or not \\
    |hp|         & number & The entity's current HP \\
    |mp|         & number & The entity's current MP \\
    |tech|       & number & The entity's current Tech \\
    |level|      & number & What level the entity is (e.g.\ 5) \\
    |experience| & number & Current experience, or the experience granted on death \\
\end{apidoc}

All numbers are integers.

Most are fairly self-explanatory.
The |experience| attribute should almost never be useful:
for players, it gives the experience they need to reach the next level.
For enemies, it gives the experience they contibute upon death.

\section{Stats}
\label{sec:entity_stats}

This is the stat block for the entity,
which can be retrieved with the |stats| \hyperref[tbl:entity_data]{entity attribute}.
This is recalculated at the start of each skill use
to account for changes in equipment, status effects, stances, and so forth.
Note this means that a skill cannot, for example,
apply a status effect to halve a target's defense
and then deal (effectively) double damage ---
the change in defense stat doesn't happen until
after the skill has finished being used.

This limitation only applies to the stat block, however ---
the entity's pools, for example, do update in ``real time''.

For example:
\begin{lstlisting}
    log(e.hp)     -- prints 10
    e:drainHP(5)  -- deal 5 damage
    log(e.hp)     -- prints 5, as expected

    log(e.stats.max_hp)  -- prints 10
    -- do something that reduces the entity's maximum health stat here
    log(e.stats.max_hp)  -- still prints 10!
\end{lstlisting}

\makeatletter
\begin{apidoc}[Entity stat block attributes][tbl:entity_stats]{lll}
    \thead{Attribute} & \thead{Type} & \thead{Description} \\
    \midrule
    |max_hp| & number & The entity's maximum HP \\
    |max_mp| & number & The entity's maximum MP \\
    |max_tech| & number & The entity's maximum Tech \\
    |p_atk| & number & The entity's physical attack power \\
    |p_def| & number & The entity's physical defensive strength \\
    |m_atk| & number & The entity's magical attack power \\
    |m_def| & number & The entity's magical defensive strength \\
    |skill| & number & The entity's accuracy modifier to increase hit chance \\
    |evade| & number & The entity's evasion modifier to decrease hit chance \\
    |speed| & number & Determines the entity's placing in the turn order\\
    \todo{Add a \lstinline{luck} attribute?}
    \midrule
    \hyperref[sec:entity_stats_resists]{\texttt{resists}} & function
              & The entity's resistance to a given \hyperref[ch:elements]{element}\\
    \note{Maybe rename \lstinline{resists} to \lstinline{affinity}?
        (And take all that that implies as well.)}
\end{apidoc}

All numbers are integers.
Note the |resists| function included in the above is not technically an attribute.
Its definition is expounded below.

\subsection{\lstinline{stats:resists(elem)}}
\label{sec:entity_stats_resists}

Calculates the entity's resistance to the given element.
This function takes one parameter, |elem|,
which is the \hyperref[ch:elements]{element} to calculate resistance for.
See \autoref{ch:elements} for a description of the available elements.

The returned value is the entity's resistance to |elem| as a percentage.
For example, an entity |e| with 20\% Fire resistance would return a value
of 20. Note that modifiers expect resistances as proportions; see below for
an example of calculating this.

\begin{lstlisting}
    -- for an entity \texttt{e} with 20\% Fire resistance, the below assert should hold.
    local resistance = e.stats:resists(element.fire)
    assert(resistance == 20)

    -- We convert this resistance into a value suitable for a modifier (0.8).
    local mod = (-resistance / 100) + 1
\end{lstlisting}

\section{Functions}
\label{sec:entity_func}

Here we list the various operations that you can perform on an entity.
For a skill's \hyperref[sec:skill_func]{\lstinline{perform} attribute}
(see \autoref{sec:skill_func}),
you may use all of these functions on both the |source| and |target| parameters.

\subsection{Pool Modifiers}
\label{sec:entity_func_pool}

All pool modifier functions have the same signature:
they take a number |amt|, which is the value to modify the pool by,
and don't return anything.
|amt| is clamped to positive numbers;
passing a negative number will have the same result as passing in |0|.
The differences between each function are described in \autoref{tbl:entity_pool}.
\begin{apidoc}[Entity pool modifiers][tbl:entity_pool]{ll}
    \thead{Function} & \thead{Description} \\
    \midrule
    |drainHP(amt)| & Reduce the entity's health by |amt| \\
    |drainMP(amt)| & Reduce the entity's mana by |amt| \\
    |drainTech(amt)| & Reduce the entity's tech by |amt| \\
    \midrule
    |restoreHP(amt)| & Restore the entity's health by |amt| \\
    |restoreMP(amt)| & Restore the entity's mana by |amt| \\
    |restoreTech(amt)| & Restore the entity's tech by |amt| \\
\end{apidoc}

None of the aforementioned functions can
raise or lower the entity's pools beyond reasonable values.
That is, you can never have health less than 0 or greater than |entity.stats.max_hp|.

\subsection{\lstinline{getTeam()}}
\label{sec:entity_func_getteam}

This function retrieves all \emph{living} team members for the entity.
It takes no parameters, and returns the entities as a list.
This list may be iterated as usual, for example:
\begin{lstlisting}
    -- get the team for the entity \texttt{source}
    local team = source:getTeam()

    -- loop over the list with \texttt{ipairs}
    -- note we ignore the first argument (the index number)
    for _, entity in ipairs(team) do
        -- use \texttt{entity} somehow; here's a nonsense example
        if entity.hp < 20 then
            log(message.notify(entity.name .. " is dying!"))
        end
    end
\end{lstlisting}

Note that this list also includes the source entity;
that is, for the call |team = source:getTeam()|
the list |team| will contain |source| as well
(well, presuming that |source| is alive, that is).

\subsection{\lstinline{getDeadTeam()}}
\label{sec:entity_func_getdeadteam}

This function is identical to \nameref{sec:entity_func_getteam}
except that instead of returning living team members
it only returns the dead ones.

\todo{Combine with \nameref{sec:entity_func_getteam}
using a (possibly optional) function parameter?}
