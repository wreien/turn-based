\chapter{Skills}
\label{ch:skill}

Skills represent things that can be done in a battle.
They are specified in Lua, and consist of
a list of key-value pairs defining their functionality.
The owner of a skill isn't necessarily an entity;
for example, skills can belong to:
\begin{itemize}[noitemsep]
    \item the entity, for learnt skills
    \item a ``stance'', for skills usable only in a particular form
    \item a consumable, for the effect of an item
    \item a piece of equipment, as a skill granted while wearing it
    \item a buff, e.g.\ the second stage of a ``charge up $\rightarrow$ attack'' skill
\end{itemize}

\section{Representation}
\label{sec:skill_representation}

To specify a skill, declare a function
taking \texttt{level} and \texttt{perks} as follows:
\begin{lstlisting}
    require "skill.base"

    skill.list["Name of Skill"] = function(level, perks)
        return {
            -- skill definition here; see below for details
        }
    end
\end{lstlisting}

Obviously, |"Name of Skill"| should be unique;
if the same skill is defined twice, the latter construction has priority.

The information passed in is as follows:

\begin{itemize}
    \item |level| represents the skill level to construct the skill at.
        It must be a value between |1| and |skill.max_level| inclusive.

    \item |perks| is a set of perks to be applied.
        You can test for the existance of a particular perk with:
\begin{lstlisting}
    if perks["power boost"] then
        -- do something
    end
\end{lstlisting}
\end{itemize}

You should return a list of key-value pairs representing
the data and functionality of the skill.

\section{Data}
\label{sec:skill_data}

Data is used both internally within the skill to ``do things'',
as well as within the calling C++ code to generate
metainfo and descriptions for a skill.
When creating a skill, some code is run
to error-check a level 1 instance of the new skill
to make sure that all the preconditions of the data is fulfilled.

\subsection{Informational}
\label{sec:skill_info}

This is the various bookkeeping required for the skill.
The attribute |desc| is required, but |max_level| can be default to |1|.
Other information may appear here in the future,
for example categorising skill types in the UI or linking to animation data.

\begin{apidoc}[Skill bookkeeping information][tbl:skill_info]{lll}
    \thead{Attribute} & \thead{Type} & \thead{Description} \\
    \midrule
    |desc| & string & An english description for the skill's effect \\
    |max_level| & number & The maximum level the skill can reach; default |1| \\
    \todo{Should we add a descriptor \lstinline|name| or something?}
\end{apidoc}

Note that |max_level| should be an integer (whole number) ---
that is, it cannot be a fractional value.

\subsection{Costs}
\label{sec:skill_costs}

Attributes used to check if the skill is castable,
and what the skill will cost the user.
All of these fields may be, and default to, |nil|,
representing no cost of that kind.

\begin{apidoc}[Skill costs][tbl:skill_costs]{lll}
    \thead{Attribute} & \thead{Type} & \thead{Description} \\
    \midrule
    |hp_cost|   & number & The amount of HP required to cast the skill \\
    |mp_cost|   & number & The amount of MP required to cast the skill \\
    |tech_cost| & number & The amount of Tech required to cast the skill \\
    |items|     & list & Any other components required to cast the skill \\
    \unimplemented{items}
\end{apidoc}


Apart from |items|, all cost values must be integers (whole numbers);
you may make use of the ``flooring'' division operator |//| to ensure this
(or otherwise use |math.floor(number)|)
if there is a possibility of creating fractional numbers
to forcefully round down to the nearest whole number.

On the other hand, |items| is a list of items required to cast the skill,
specified as strings representing the name of the item;
specify the same item multiple times as necessary, as in:
\begin{lstlisting}
    -- require two herbs and a flower to cast the skill
    items = { "herb", "herb", "flower" }
\end{lstlisting}

\subsection{Attributes}
\label{sec:skill_attrs}

There are a variety of other attributes required to specify
the high-level behaviour of the skill.
Unless otherwise specified, these default to |nil|, a ``not applicable'' value.
For example, a skill to poison an enemy would have |power = nil|,
since it has no ``power'' \emph{per se},
and could thus leave it out.
However, said skill should still specify |accuracy = 100| (unless it can't miss).

\begin{apidoc}[General skill attributes][tbl:skill_attrs]{lll}
    \thead{Attribute} & \thead{Type} & \thead{Description} \\
    \midrule
    |power| & number & The base damage for the skill \\
    |accuracy| & number & The base percentage chance for the skill to hit \\
    |method| & \hyperref[tbl:opt_method_spread]{method}
             & What stats of the source it uses; default |method.none| \\
    |spread| & \hyperref[tbl:opt_method_spread]{spread}
             & Who the skill targets; default |spread.single| \\
    |element| & \hyperref[ch:elements]{element}
              & What element the skill is; default |element.neutral| \\
\end{apidoc}
\begin{apidoc}[Options for method and spread][tbl:opt_method_spread]{ll}
    \todo{Probably the attribute names and type names should be different in some way;
    maybe some kind of namespace for the types?}
    \thead{Value}     & \thead{Description} \\
    \midrule
    |method.physical| & Deals primarily physical damage (uses |p_atk| and |p_def|) \\
    |method.magical|  & Deals primarily magical damage (uses |m_atk| and |m_def|) \\
    |method.mixed|    & Deals damage that's neither primarily physical nor magical \\
    |method.none|     & Doesn't deal direct damage \\
    \midrule
    |spread.self|     & Can only target the skill user, i.e.\ source = target \\
    |spread.single|   & Targets any one entity \\
    |spread.aoe|      & Targets an entire team equally \\
    |spread.semiaoe|  & Targets an entire team, but focusses on an individual \\
    |spread.field|    & Targets the entire battlefield \\
    \unimplemented{spread.field}
\end{apidoc}

Like with the \hyperref[tbl:skill_costs]{costs},
|power| and |accuracy| should be integers.
Additionally, an |element| is any one of the available elements.
See \autoref{ch:elements} for more details.

\section{Functionality}
\label{sec:skill_func}

To define exactly what the skill does, an attribute |perform| must be provided.
This is a function taking three parameters:
\begin{itemize}[noitemsep]
    \item |self| (or |s|): me, the skill using the function (and its details)
    \item |source|: whomever used the skill in the first place
    \item |target|: the entity the skill targeted (or |nil| if |spread.field|)
\end{itemize}

(See also \autoref{tbl:func_skill_params}.)

The parameter |self| (or |s|) contains
all the data specified in the skill you returned;
see \autoref{sec:skill_data} for details.
For example, if you returned a skill with 80 power, then |self.power| is 80.
Prefer to use this value rather than recalculating the desired attributes.
The only calculated values you might ever need are |level| and |perks|,
as described in \autoref{sec:skill_representation}.

The parameters |source| and (if it exists) |target| are \hyperref[ch:entities]{entities}.
They are described in detail in \autoref{ch:entities}.

While developing, a useful placeholder for the |perform| attribute's function is
|skill.default_perform|, which guesses a standard implementation
based upon the skill's other attributes.
See \hyperref[ch:func]{the helper function documentation}
(\autoref{ch:func}) for more details.

\section{Example}
\label{sec:skill_example}

Here is a detailed example of a skill,
demonstrating usage of most of the attributes described above.

\begin{lstlisting}
    -- we define a skill named "Eruption"
    skill.list["Eruption"] = function(level, perks)
        -- we set up some defaults
        local power = 50 + 15 * level
        local accuracy = 80
        -- we call it newspread rather than spread
        -- so we can still access the global enum ``spread'' later
        local newspread = spread.semiaoe

        -- we manage our perks
        if perks["overheat"] then
            -- make sure we end up with a whole number!
            power = math.floor(power * 1.2)
            accuracy = accuracy - 10
        end
        if perks["even spread"] then
            newspread = spread.aoe
        end

        -- now we actually provide the data for our skill
        return {
            -- for niceness, we can split the long description over multiple lines
            desc = "Unleash a volcanic eruption at a target, "
                .. "damaging their whole team.",
            max_level = 5,

            -- we leave out \texttt{hp\_cost} here as it's not relevant
            mp_cost = 30,
            tech_cost = 10,
            -- we need two fire stones to use the skill
            items = { "fire stone", "fire stone" },

            -- pass in our calculated values
            power = power,
            accuracy = accuracy,
            spread = newspread,
            method = method.magical,
            element = element.fire,

            -- provide a default function to actually do things
            perform = function(self, source, target)
                -- we can use perks inside the perform function, too
                if not perks["consistency"] then
                    if random(10) == 1 then
                        log(message.notify("But it failed!"))
                        return -- quit early
                    end
                end

                -- delegate to a different perform function
                skill.default_perform(self, source, target)
            end
        }
    end
\end{lstlisting}
